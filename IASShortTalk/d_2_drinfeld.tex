\section{Drinfeld modular forms}


\subsection{Set the stage}

\begin{frame} \frametitle{The Drinfeld setting}
  We now move from the classical setting to that of function field arithmetic.
  % Most of the data are analogous with more familiar objects:

  \textbf{Function field object} \hfill \textbf{Classical analogue} \pause \\
  $F$, a fixed global function field \hfill $\QQ$, the rational numbers \pause \\
  $\abs{\cdot}$, a fixed absolute value on $F$ with associated place $\infty$ \hfill the usual absolute value $\abs{\cdot}$ on $\CC$ \pause \\
  $A$, the ring of elements of $F$ regular away from $\infty$ \hfill $\ZZ$, the integers \pause \\
  $\FF_\infty$, the completion of $F$ with respect to $\abs{\cdot}$ \hfill $\RR$, the real numbers \pause \\
  $\CC_\infty$, the completion of an algebraic closure of $\FF_\infty$ \hfill $\CC$, the complex numbers \pause

  Here, analogously with the classical setting, $A$ is a Dedekind ring.
  We also consider the positive integer $q$, which is the cardinality of the field of constants of $F$, with associated finite field $\FF_q$.
\end{frame}


\subsection{The only constant is change}

\begin{frame} \frametitle{Differences to the classical setting}
  In contrast with the classical setting, there are some key differences: \pause
  \begin{itemize}
    \item The main rings are of finite characteristic. \pause
    \item The absolute value $\abs{\cdot}$ is \emph{non-archimedean}; hence analytic issues such as convergence of series are in some ways easier, whereas defining an analytic function is more complex than in the classical case. \pause
    % \item As previously mentioned, lattices can have arbitrarily high integer rank. \pause
    \item In general, the elements of $A$ do not uniquely factorise into products of prime elements. However, since $A$ is a Dedekind ring we do have unique factorisation of \emph{ideals} into products of prime ideals. \pause So we henceforth let $N$ be an arbitrary proper ideal of $A$.
    \item Here $\CC_\infty$ has infinite dimension as a vector space over $\FF_\infty$, whereas $\CC$ has dimension $2$ as a vector space over $\RR$. \pause
    As a result, whereas lattices in the classical case can have rank at most $2$, here lattices can have arbitrarily high rank. This what makes the theory of `modular forms of higher rank' possible.
  \end{itemize}
\end{frame}


\subsection{What am I even doing?}

\begin{frame} \frametitle{My work}
  Whereas most of the theory of Drinfeld modular forms has focused on modular forms of rank $2$, due to the simplicity of dealing with a function of one variable, recent work by Gekeler and Basson, Breuer, and Pink have established theories of Drinfeld modular forms of higher rank. \pause
  
  Their approach has been mainly as viewing them as functions of $r-1$ variables, whereas my PhD thesis established a theory viewing them as functions on the space of lattices of higher rank. \pause
  My current work in this area is on a theory of Hecke operators from this point of view.

  Additionally, in the rank 2 case there is the question of generators for the (graded) algebra of modular forms for the principal congruence subgroup $\Gamma(N)$, where I have some partial computational results.
\end{frame}


\begin{frame} \frametitle{Lattices}
  \begin{definition}
    A lattice $\Lambda$ of rank $r$ is a projective $A$-submodule of $\CC_\infty$ of rank $r$ \pause (\ie a subset of $\CC_\infty$ of the form $I_1\psi_1 +\dotsb +I_r\psi_r$ for ideals $I_i \subseteq A$ and $\psi_i \in \CC_\infty$ which are $\FF_\infty$-linearly independent). \pause

    A level $N$ structure for a lattice $\Lambda$ of rank $r$ is an $A$-module bijection $\parens{N^{-1}/A}^r \ionto N^{-1}\Lambda/\Lambda$.
  \end{definition} \pause

  Every lattice has $\size{\GL[r]{A/N}}$ different level $N$ structures, since $A$ is a Dedekind ring.

  We denote the space of lattices $\Lambda$ of rank $r$ with level $N$ structure $\alpha$ by $\LL_N^r$, and the space of lattices $\Lambda$ of rank $r$ without level structure by $\LL^r$. \pause
  These are rigid analytic spaces, with left actions of $\gamma \in \GL[2]{A/N}$ and fractional ideals $J \in \cJ(A)$ given by
  \[ \gamma(\Lambda,\alpha) \defeq (\Lambda, \alpha \circ \gamma^{-1}) \lrsptext{and} J.\Lambda \defeq J^{-1} \Lambda. \]
\end{frame}


\begin{frame} \frametitle{Lattices II}
  We define metrics $\dd_\LL$ and $\dd_{\LL_N^r}$ on the spaces $\LL^r$ and $\LL_N^r$, leading to their completions $\LL^{\leq r}$ and $\LLNRi$, where the boundaries consist of spaces of lattices of lower rank. \pause
  \begin{definition} \label{def:strongMForm}
    A \emph{modular form} of \emph{weight} $k$ and \emph{rank} $r$ for $K(N)$ is a function $f : \LLNRi \to \CCi$ which is: \pause
    \begin{itemize}
      \item holomorphic on the interior $\LL_N^r$ of $\LLNRi$\pause,
      \item homogeneous of degree $-k$\pause, and
      \item continuous on $\LLNRi$. \pause
    \end{itemize}
  \end{definition}

  The left action of $\gamma \in \GL[2]{A/N}$ on $\LLNRi$ translates to a right action on the space of modular forms, by $f|\gamma(\Lambda,\alpha) = f(\Lambda, \alpha \circ \gamma^{-1})$.
\end{frame}