\documentclass[hyperref = {unicode},aspectratio = 169]{beamer}
\mode<presentation>
\usepackage[utf8]{inputenc}				% native UTF-8 characters in file
\usepackage{amssymb}							% maths symbols
\usepackage{mathtools}						% loads and improves amsmath
\usepackage{amsthm}								% theorems
\usepackage{thmtools,thm-restate}	% theorem improvements
\usepackage{graphicx}							% loading pictures
\usepackage{silence}							% suppressing warnings
\usepackage{tikz-cd}							% commutative diagrams
\usetikzlibrary{babel}						% allows maths inside tikzcd env
\usepackage{parskip}							% space between paragraphs
\usepackage{mleftright}						% slimmer auto-delimiter spacing

\hypersetup{
	% colorlinks = true,
	linkcolor = blue,
	citecolor = green,
	urlcolor = black
}

\declaretheorem{proposition}

\setlength{\parskip}{6pt plus 2pt minus 2pt}

\graphicspath{{./images/}}

% use special characters instead of numbers for footnotemarks:
\renewcommand{\thefootnote}{\fnsymbol{footnote}}

\hfuzz = 5pt
\newdimen\hfuzz	% for ignoring small overfull hbox's

\definecolor{uibred1}{RGB}{180, 50, 100}
\definecolor{uibred}{RGB}{100, 0, 0}
\definecolor{uibblue}{RGB}{0, 84, 115}
\definecolor{uibgreen}{RGB}{119, 175, 0}
\definecolor{uibgreen1}{RGB}{50, 105, 0}
\definecolor{uiborange}{RGB}{217, 89, 0}

\setbeamertemplate{blocks}[rounded][shadow = false]
\addtobeamertemplate{block begin}{\pgfsetfillopacity{0.8}}{\pgfsetfillopacity{1}}
\setbeamercolor{structure}{fg = uibred}
\setbeamercolor*{block title example}{fg = white,bg = uibred}
\setbeamercolor*{block body example}{fg = black,bg = uibred!10}

%% custom commands
% primed operators
% TeXbook 18.44
\def\p[#1]_#2{
	\setbox0 = \hbox{$\scriptstyle{#2}$}
	\setbox2 = \hbox{$\displaystyle{#1}$}
	\setbox4 = \hbox{${}'\mathsurround = 0pt$}
	\dimen0 = .5\wd0 \advance\dimen0 by-.5\wd2
	\ifdim\dimen0>0pt
	\ifdim\dimen0>\wd4 \kern\wd4 \else\kern\dimen0\fi\fi
	\mathop{{#1}'}_{\kern-\wd4 #2}
}
\def\sump_#1{\p[\sum]_{#1}}
\def\prodp_#1{\p[\prod]_{#1}}
\def\minp_#1{\p[\min]_{#1}}
\def\maxp_#1{\p[\max]_{#1}}
% sets
\newcommand*{\NN}{\mathbb{N}}
\newcommand*{\NNO}{\mathbb{N}_0}
\newcommand*{\ZZ}{\mathbb{Z}}
\newcommand*{\QQ}{\mathbb{Q}}
\newcommand*{\RR}{\mathbb{R}}
\newcommand*{\CC}{\mathbb{C}}
\newcommand*{\CCi}{\mathbb{C}_\infty}
\newcommand*{\HH}{\mathcal{H}}
\newcommand*{\FF}{\mathbb{F}}
\newcommand*{\PP}{\mathbb{P}}
\newcommand*{\LL}{\mathcal{L}}
\renewcommand{\AA}{\mathbb{A}}
\newcommand*{\cJ}{\mathcal{J}}
\newcommand*{\finadele}{\AA_F^{fin}}
\newcommand*{\invertadele}{\bigl(\finadele\bigr)^{\!\times}}
\newcommand*{\fp}{\mathfrak{p}}
\newcommand*{\WeakMF}{\mathfrak{W}}
\newcommand*{\StrongMF}{\mathfrak{M}}
\newcommand*{\CuspMF}{\mathfrak{S}}
\newcommand*{\cw}{\mathcal{W}}
\newcommand*{\cm}{\mathcal{M}}
\newcommand*{\cs}{\mathcal{S}}
\newcommand*{\LLi}[2]{\protect\overleftarrow{\LL_{#1}^{#2}}}
\newcommand*{\LLNRi}{\LLi{N}{r}}
% arrows
% \newcommand*{\isoto}{\overset{\sim}{\to}}
\newcommand*{\isoto}{\xrightarrow{\mathmakebox[1.5ex]{\sim}}}
\newcommand*{\into}{\hookrightarrow}
\newcommand*{\onto}{\twoheadrightarrow}
\newcommand*{\ionto}{\lhook\joinrel\twoheadrightarrow}
\newcommand*{\xinto}[2][]{\xhookrightarrow[#1]{#2}}
\newcommand*{\xonto}[2][]{\xrightarrow[#1]{#2}\mathrel{\mkern-14mu}\rightarrow}
\newcommand*{\xionto}[2][]{\xhookrightarrow[#1]{#2}\mathrel{\mkern-22mu}\rightarrow\mathrel{\mkern6mu}}
\newcommand*{\mapsfrom}{\mathrel{\reflectbox{\ensuremath{\mapsto}}}}
% maths operators
\DeclareMathOperator{\End}{End}
\DeclareMathOperator{\B}{B}
\DeclareMathOperator{\D}{D}
\DeclareMathOperator{\Cl}{Cl}
\let\Im = \relax \DeclareMathOperator{\Im}{Im}
\DeclareMathOperator{\Sur}{Sur}
\DeclareMathOperator{\Inj}{Inj}
\DeclareMathOperator{\Free}{Free}
\DeclareMathOperator{\GLL}{GL}
\newcommand*{\GL}[2][]{\GLL_{#1}\parens{#2}}
\DeclareMathOperator{\SLL}{SL}
\newcommand*{\SL}[2][]{\SLL_{#1}\parens{#2}}
\DeclareMathOperator{\Div}{Div}
\DeclareMathOperator{\bd}{\partial}
% differentiation
\newcommand*{\dd}{\operatorname{d}}
\newcommand*{\id}[1]{\,\dd{#1}}
\newcommand*{\dbd}[2][]{\frac{\dd^{#1}}{\dd\!#2^{#1}}}
% enclosures
\mleftright	% fixes spacing around left-(middle-)right delimiters
% \delimiterfactor = 851
\delimitershortfall = 10pt
\newcommand*{\set}[1]{\left\{#1\right\}}
\newcommand*{\setst}[2]{\left\{#1\:\middle|\:#2\right\}}
\newcommand*{\order}[1]{\!O\!\left(#1\right)\!}
\newcommand*{\norm}[1]{\left\lVert 1\right\rVert}
\newcommand*{\abs}[1]{\left\lvert#1\right\rvert}
\newcommand*{\floor}[1]{\left\lfloor#1\right\rfloor}
\newcommand*{\ceil}[1]{\left\lceil#1\right\rceil}
\newcommand*{\fracpart}[1]{\left\{#1\right\}}
\newcommand*{\parens}[1]{\left(#1\right)}
\newcommand*{\pparens}[1]{\parens{\!\parens{#1}\!}}
\newcommand*{\brackets}[1]{\left[#1\right]}
\newcommand*{\braces}[1]{\left\{#1\right\}}
\newcommand*{\size}{\#}
\newcommand*{\sizep}[1]{\size\parens{#1}}
\newcommand*{\sizeset}[1]{\size\set{#1}}
\newcommand*{\sizesetst}[2]{\size\setst{#1}{#2}}
\newcommand*{\lquotient}[2]{\left. #1 \middle\backslash #2 \right.}
\newcommand*{\rquotient}[2]{\left. #1 \middle/ #2 \right.}
\newcommand*{\lrquotient}[3]{\left. #1 \middle\backslash #2 \middle/ #3 \right.}
\newcommand*{\cin}[2]{\bigl[{#1}^{#2}\bigr]}
\newcommand*{\bigO}[1]{\mathcal{O}\parens{#1}}
\newcommand*{\littleo}[1]{o\parens{#1}}
\newcommand*{\pcoord}[1]{%
  \begingroup\lccode`~ = `: \lowercase{\endgroup
  \edef~}{\mathbin{\mathchar\the\mathcode`:}\nobreak}%
  \left(% opening symbol
  \begingroup
  \mathcode`: = \string"8000
  #1%
  \endgroup
  \right)% closing symbol
}
% maths text spacing
\newcommand*{\lrsptext}[1]{\quad\text{#1}\quad}
\newcommand*{\lsptext}[1]{\quad\text{#1}\ }
\newcommand*{\rsptext}[1]{\ \text{#1}\quad}
\newcommand*{\sptext}[1]{\ \text{#1}\ }
% other
\newcommand*{\ie}{i.e.\ }
\newcommand*{\eqtag}{\stepcounter{equation} \tag{\theequation}}
\newcommand*{\defeq}{\coloneqq}
\newcommand*{\eqdef}{\eqqcolon}



\mode<presentation>{
  \usetheme{ift}
  %\setbeamercovered{transparent}	% commented because it messes with uncovering lines in align environments
  \setbeamertemplate{items}[square]
}

\usefonttheme[onlymath]{serif}
\setbeamerfont{frametitle}{size = \Large}
\setbeamertemplate{navigation symbols}{}
% \AtBeginSection[]{
% 	\begin{frame} \vfill \centering
% 	\begin{beamercolorbox}[sep = 8pt,center]{title}
% 		\usebeamerfont{title}Chapter~\thesection \\\vspace{12pt} \insertsectionhead\par
% 	\end{beamercolorbox}
% 	\vfill \end{frame}
% }

\title[Drinfeld modular forms]
  {\LARGE Drinfeld modular forms}
\subtitle{\large IAS Short Talk}
\author[Liam Baker]{Liam Baker}
\date[2023/10/05]{5 October 2023}
\institute[Stellenbosch University]{Department of Mathematical Sciences \\ Stellenbosch University}


\begin{document}

\setbeamertemplate{background}{
  \includegraphics[width = \paperwidth, height = \paperheight]{frontpage_bg_169}
}
\setbeamertemplate{footline}[default]
\begin{frame}
  \titlepage
\end{frame}


\setbeamertemplate{background}{
  \includegraphics[width = \paperwidth, height = \paperheight]{slide_bg_169}
}
\setbeamertemplate{footline}[ifttheme]


\section{Classical Modular Forms}


\subsection{These are pretty cool!}

\begin{frame} \frametitle{Classical modular forms -- connections}
  Modular forms are analytic functions on the complex upper half plane which have important number theoretical properties. % moar
\end{frame}


\begin{frame} \frametitle{Classical modular forms -- Definition 1} \pause
  \begin{definition}
    A (classical) modular form $f$ of weight $k \in \NNO$ for the group $\SL[2]{\ZZ}$ is a function on the complex upper half plane $\HH = \setst{z \in \CC}{\Im{z} > 0}$ satisfying the following properties: \pause
    \begin{itemize}
      \item $f$ is complex analytic on $\HH$\pause,
      \item $f\parens{\frac{az+b}{cz+d}} = (cz+d)^{k} f(z)$ for all $a,b,c,d \in \ZZ$ such that $ad-bc = 1$\pause, and
      \item $\abs{f(z)}$ is bounded as $\Im{z} \to +\infty$.\pause \hfill \emph{(holomorphic at infinity)}
    \end{itemize}
  \end{definition}

  The second condition can be written as $f(\gamma z) = j(\gamma,z)^{-k} f(z)$ for all $\gamma = \parens{\begin{smallmatrix} a & b \\ c & d \end{smallmatrix}} \in \SL[2]{\ZZ}$, with \emph{factor of automorphy} $j(\gamma,z) = cz+d$.
\end{frame}


\begin{frame} \frametitle{Classical modular forms -- Definition 1}
  Here the group $\SL[2]{\ZZ}$ acts on $\HH$ from the left by the fractional linear transformation $z \xmapsto{\gamma} \frac{az+b}{cz+d}$, which is closely related to matrix multiplication on the left:
  \[ \begin{pmatrix} a & b \\ c & d \end{pmatrix} \begin{pmatrix} z \\ 1 \end{pmatrix} = \begin{pmatrix} az+b \\ cz+d \end{pmatrix} = \frac{1}{cz+d} \begin{pmatrix} \frac{az+b}{cz+d} \\ 1 \end{pmatrix}. \]
  \pause
  The left action of $\SL[2]{\ZZ}$ on $\HH$ translates into a right action on the functions $f : \HH \to \CC$:
  \[ f|_\gamma : \HH \to \CC, \quad z \mapsto j(\gamma,z)^{-k} f(\gamma z) = (cz+d)^{-k} f\parens{\frac{az+b}{cz+d}}. \]
  \pause
  The transformation condition in the previous definition can then be restated more simply: \pause
  \[ f|_\gamma(z) = f(z) \lsptext{for all} \gamma \in \SL[2]{\ZZ}. \]
\end{frame}


\begin{frame} \frametitle{Classical modular forms -- Definition 1}
  The first definition above is that of a modular form for the \emph{full} group $\SL[2]{\ZZ}$.
  More generally, for any $N \in \NN$ we can define a modular form for any \emph{congruence subgroup} $\Gamma \subseteq \SL[2]{\ZZ}$:
  \[ \Gamma \supseteq \Gamma(N) = \setst{\parens{\begin{smallmatrix} a & b \\ c & d \end{smallmatrix}} \in \SL[2]{\ZZ}}{\parens{\begin{smallmatrix} a & b \\ c & d \end{smallmatrix}} \equiv \parens{\begin{smallmatrix} 1 & 0 \\ 0 & 1 \end{smallmatrix}} \pmod{N}}: \pause \]
  \begin{definition}
    A modular form $f$ of weight $k$ for the congruence subgroup $\Gamma$ is a function $f : \HH \to \CC$ such that: \pause
    \begin{itemize}
      \item $f$ is complex analytic on $\HH$\pause,
      \item $f|_\gamma(z) = f(z)$ for all $\gamma \in \Gamma$\pause, and
      \item $\abs{f|_\gamma(z)}$ is bounded as $\Im{z} \to +\infty$ for all $\gamma \in \SL[2]{\ZZ}$. \pause \hfill \emph{(holomorphic at the cusps)}
    \end{itemize}
  \end{definition} \pause

  % Note that if $\Gamma = \Gamma(1) = \SL[2]{\ZZ}$, we get a modular form for the full group $\SL[2]{\ZZ}$.
\end{frame}


\subsection{A function on lattices?}

\begin{frame} \frametitle{Classical modular forms -- Definition 2}
  An alternative way to define a modular form is as a function on the space $\LL$ of lattices of rank $2$\pause; here a lattice (of rank $2$) is a free rank-$2$ additive subgroup $\Lambda = a\ZZ +b\ZZ \subset \CC$:\pause

  \includegraphics[width = \textwidth]{lattice.pdf}
\end{frame}


\begin{frame} \frametitle{Classical modular forms -- Definition 2}
  \begin{definition}
    A modular form $f$ of weight $k$ for $\SL[2]{\ZZ}$ is a function $f : \LL \to \CC$ such that:
    \begin{itemize}
      \item $f$ is analytic\pause, \hfill (?) \pause
      \item $f$ is homogeneous of degree $-k$: \pause $f(r\Lambda) = r^{-k} f(\Lambda)$ for all $r \in \CC^\times$ and $\Lambda \in \LL$\pause, and
      \item $\abs{f(\Lambda)}$ is bounded as long as the smallest element of $\Lambda$ is bounded away from $0$.\pause
    \end{itemize}
  \end{definition}

  If $f$ is a modular form in this lattice sense, then $\bar{f}(z) = f(z\ZZ+\ZZ)$ is a modular form in the complex-variable sense.
  In fact, the converse is also true, so that these two definitions are equivalent. \pause

  To define a modular form for the congruence subgroup $\Gamma(N)$ as a function of lattices we actually define it as a homogeneous function $f(\Lambda,\alpha)$ of a lattice $\Lambda$ together with a \emph{level structure} $\alpha : N^{-1}\Lambda/\Lambda \ionto (N^{-1}/\ZZ)^2$.
\end{frame}

\section{Drinfeld modular forms}


\subsection{Set the stage}

\begin{frame} \frametitle{The Drinfeld setting}
  We now move from the classical setting to that of function field arithmetic.
  % Most of the data are analogous with more familiar objects:

  \textbf{Function field object} \hfill \textbf{Classical analogue} \pause \\
  $F$, a fixed global function field \hfill $\QQ$, the rational numbers \pause \\
  $\abs{\cdot}$, a fixed absolute value on $F$ with associated place $\infty$ \hfill the usual absolute value $\abs{\cdot}$ on $\CC$ \pause \\
  $A$, the ring of elements of $F$ regular away from $\infty$ \hfill $\ZZ$, the integers \pause \\
  $\FF_\infty$, the completion of $F$ with respect to $\abs{\cdot}$ \hfill $\RR$, the real numbers \pause \\
  $\CC_\infty$, the completion of an algebraic closure of $\FF_\infty$ \hfill $\CC$, the complex numbers \pause

  Here, analogously with the classical setting, $A$ is a Dedekind ring.
  We also consider the positive integer $q$, which is the cardinality of the field of constants of $F$, with associated finite field $\FF_q$.
\end{frame}


\subsection{The only constant is change}

\begin{frame} \frametitle{Differences to the classical setting}
  In contrast with the classical setting, there are some key differences: \pause
  \begin{itemize}
    \item The main rings are of finite characteristic. \pause
    \item The absolute value $\abs{\cdot}$ is \emph{non-archimedean}; hence analytic issues such as convergence of series are in some ways easier, whereas defining an analytic function is more complex than in the classical case. \pause
    % \item As previously mentioned, lattices can have arbitrarily high integer rank. \pause
    \item In general, the elements of $A$ do not uniquely factorise into products of prime elements. However, since $A$ is a Dedekind ring we do have unique factorisation of \emph{ideals} into products of prime ideals. \pause So we henceforth let $N$ be an arbitrary proper ideal of $A$.
    \item Here $\CC_\infty$ has infinite dimension as a vector space over $\FF_\infty$, whereas $\CC$ has dimension $2$ as a vector space over $\RR$. \pause
    As a result, whereas lattices in the classical case can have rank at most $2$, here lattices can have arbitrarily high rank. This what makes the theory of `modular forms of higher rank' possible.
  \end{itemize}
\end{frame}


\subsection{What am I even doing?}

\begin{frame} \frametitle{My work -- higher rank}
  Whereas most of the theory of Drinfeld modular forms has focused on modular forms of rank $2$, due to the simplicity of dealing with a function of one variable, recent work by Gekeler (for $\FF_q(T)$) and Basson, Breuer, and Pink (for general $F$) have established theories of Drinfeld modular forms of higher rank. \pause
  
  Their approach has been viewing them as functions of $r-1$ variables, whereas my PhD thesis established a theory viewing them as functions on the space of lattices of higher rank. \pause
  My current work in this area is filling out the theory from this point of view.

  % Additionally, in the rank 2 case there is the question of generators for the (graded) algebra of modular forms for the principal congruence subgroup $\Gamma(N)$, where I have some partial computational results.
\end{frame}


\begin{frame} \frametitle{Lattices}
  \begin{definition}
    A lattice $\Lambda$ of rank $r$ is a projective $A$-submodule of $\CC_\infty$ of rank $r$ \pause (\ie a subset of $\CC_\infty$ of the form $I_1\psi_1 +\dotsb +I_r\psi_r$ for ideals $I_i \subseteq A$ and $\psi_i \in \CC_\infty$ which are $\FF_\infty$-linearly independent). \pause

    A level $N$ structure for a lattice $\Lambda$ of rank $r$ is an $A$-module bijection $\parens{N^{-1}/A}^r \ionto N^{-1}\Lambda/\Lambda$.
  \end{definition} \pause

  Every lattice has $\size{\GL[r]{A/N}}$ different level $N$ structures, since $A$ is a Dedekind ring.

  We denote the space of lattices $\Lambda$ of rank $r$ with level $N$ structure $\alpha$ by $\LL_N^r$, and the space of lattices $\Lambda$ of rank $r$ without level structure by $\LL^r$. \pause
  % The left action of $\gamma \in \GL[2]{A/N}$ on $\LL_N^r$ is given by
  % \[ \gamma(\Lambda,\alpha) \defeq (\Lambda, \alpha \circ \gamma^{-1}). \]

  We prove that these spaces are rigid analytic spaces by identifying them with a double quotient:
  \[ \LL_N^r \simeq \left. \GL[r]{F} \middle\backslash \parens{\Psi^r \times \GL[r]{\AA_F^{fin}}/K(N)} \right. \]
  and so we can speak of holomorphic functions on these spaces.
\end{frame}


\begin{frame} \frametitle{Higher rank modular forms} \pause
  We define metrics $\dd_\LL$ and $\dd_{\LL_N^r}$ on the rigid analytic spaces $\LL^r$ and $\LL_N^r$, leading to their completions $\LL^{\leq r}$ and $\LLNRi$, where the boundaries consist of spaces of lattices of lower rank. \pause
  \begin{definition} \label{def:strongMForm}
    A \emph{modular form} of \emph{weight} $k$ and \emph{rank} $r$ for $K(N)$ is a function $f : \LLNRi \to \CCi$ which is: \pause
    \begin{itemize}
      \item holomorphic on the interior $\LL_N^r$ of $\LLNRi$\pause,
      \item homogeneous of degree $-k$\pause, and
      \item continuous on $\LLNRi$. \pause
    \end{itemize}
  \end{definition}

  % The left action of $\gamma \in \GL[2]{A/N}$ on $\LLNRi$ translates to a right action on the space of modular forms, by $f|_\gamma(\Lambda,\alpha) = f(\Lambda, \alpha \circ \gamma^{-1})$.

  My PhD thesis was on the general theory of modular forms of higher rank as functions of lattices, and I am currently extending this theory in the following areas: \pause
  \begin{itemize}
    \item Proving that modular forms have Fourier-type series expansions at cusps \pause
    \item Defining Hecke operators and proving their recursive properties
  \end{itemize}
\end{frame}


\begin{frame} \frametitle{My work -- Eisenstein series in rank $2$} \pause
  Typical examples of modular forms of rank $2$ for $\Gamma(N)$ are the (partial) Eisenstein series:
  \[ E_{r_1,r_2}(z) = \sum_{m,n \in A} \frac{1}{(m+r_1)z+n+r_2} \lsptext{for} r_1,r_2 \in N^{-1}A/A; \]
  here we are considering the case of $A = \FF_q[T]$. \pause

  Cornelissen proved that the graded algebra of Drinfeld modular forms of rank $2$ for $\Gamma(N)$ are generated by these Eisenstein series and possibly some cusp forms of weight $2$, but it is not known whether or not these cusp forms are necessary. \pause
  
  I have some partial computational results in this direction: for specific $N$ we can reduce it to linear algebra using the series expansions of these Eisenstein series at the cusps and the known dimension of the space of weight $2$ modular forms, but I hope to finish it off analytically.
\end{frame}
  
\end{document}