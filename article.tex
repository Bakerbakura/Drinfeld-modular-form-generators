\documentclass{article}

\usepackage{parskip}
\usepackage{mathtools}
\usepackage{fullpage}

\title{The Generation of Rank 2 Drinfeld Modular Forms by Eisenstein series: A Computational Approach}
\author{Liam Baker}
\date{\today}


\begin{document} \maketitle

\begin{abstract}
  A computational method is presented for determining whether or not the space of Drinfeld modular forms for a principal congruence subgroup $\Gamma(N)$ is generated by Eisenstein series, which is currently an open question.
  This method boils down to expressing the dimension of the weight $2$ space generated by Eisenstein series as the rank of a matrix of coefficients of products of Eisenstein series.
  As a result, the question is answered in the case of $q = 2$ and $N$ quadratic or cubic and the case of $q = 3$ and $N$ quadratic.
\end{abstract}


\section{Introduction and Outline}

Drinfeld modules are analogous to elliptic curves in the function field setting, and were first defined by <Drinfeld>, who called them \emph{elliptic modules} and used them to prove <Langlands>.
Later, <Goss> defined what he called \emph{Drinfeld modular forms} of rank $2$ analogously with classical modular forms (in this paper, we will call Drinfeld modular forms simply modular forms, and explicitly mention when their classical counterparts are named).
These functions form a graded algebra graded by their \emph{weight}.
Goss also imposed a condition of `boundedness at infinity' which allowed him to show the finite-dimensionality of the spaces of D-modular forms of any given weight.
<Gekeler> defined Eisenstein series as cases of modular forms, and showed that Eisenstein series of higher weight are generated by Eisenstein series of weight $1$.
The question then naturally arose of whether or not the space of all modular forms is generated as an algebra by the Eisenstein series of weight $1$; this was answered in the affirmative by <Cornellissen> for linear principal congruence subgroups, but the question for general $N$ is still open.
The most significant progress thus far has been by <someone>, who showed that the space of all modular forms is generated by the Eisenstein series of weight $1$, possibly together with some cusp forms of weight $2$ (these are modular forms which are zero at all the cusps).
It thus suffices to establish whether or not the Eisenstein series of weight $1$ generates the space of modular forms of weight $2$, and in this paper we give a method for doing so for any specific characteristic $q$ and principal congruence subgroup $\Gamma(N)$, as well as an implementation of this method in SageMath.

Let $q$ be a prime power, and let $\FF_q$ denote the finite field of cardinality $q$.

\section{Conclusion}

Due to computational constraints we are only able to run the code for some small nonlinear values of $N$, but we hope that this progress will inspire others to improve on our method or to revisit this interesting question with fresh eyes.

\end{document}