\documentclass[11pt]{article}

\usepackage{parskip}
\usepackage{amssymb}
\usepackage{mathtools}
\usepackage{fullpage}
\usepackage{amsthm}
% \usepackage{chngcntr}							% counter manipulation
\usepackage{thmtools,thm-restate}	% theorem improvements
\usepackage[unicode]{hyperref}		% for nice hyperlinks
\usepackage[english,capitalise,nameinlink,noabbrev]{cleveref}

\hypersetup{
	colorlinks=true,
	linkcolor=blue,
	citecolor=green,
	urlcolor=black
}

% use special characters instead of numbers for footnotemarks:
\renewcommand{\thefootnote}{\fnsymbol{footnote}}

\newcounter{common}[section]
\makeatletter
	\let\c@equation=\c@common
	\let\c@subsubsection=\c@common
	\let\c@paragraph=\c@common
\makeatother
% make these not reset at each new (sub)(sub)section
\counterwithout{subsection}{section}
\counterwithout{subsubsection}{subsection}
\counterwithout{paragraph}{subsubsection}

% theorem-like environments
\declaretheoremstyle[
	spaceabove=1.5\parskip, spacebelow=0.5\parskip,
	headfont=\normalfont\bfseries, bodyfont=\itshape,
	notefont=\mdseries, notebraces={(}{)},
	headformat=margin, postheadspace={ }, headpunct={.}
]{myplain}
\declaretheorem[sibling=common, style=myplain]{lemma}
\declaretheorem[sibling=common, style=myplain]{theorem}
\declaretheorem[sibling=common, style=myplain]{proposition}
\declaretheorem[sibling=common, style=myplain]{corollary}
\declaretheoremstyle[
	spaceabove=1.5\parskip, spacebelow=0.5\parskip,
	headfont=\normalfont\bfseries, bodyfont=\normalfont,
	notefont=\mdseries, notebraces={(}{)},
	headformat=margin, postheadspace={ }, headpunct={.}
]{mydef}
\declaretheorem[sibling=common, style=mydef]{definition}
\declaretheorem[sibling=common, style=mydef]{example}

% moving equation tags into the margin
\newtagform{margin}[\llap]{\hspace{-4pt}}{}
\usetagform{margin}

% redefining proof environment for less space before
\expandafter\let\expandafter\oldproof\csname\string\proof\endcsname
\let\oldendproof\endproof
\renewenvironment{proof}[1][\proofname]{%
  \vspace{-1.5\parskip} \oldproof[#1]%
}{\oldendproof}

%% custom commands
% primed operators
% TeXbook 18.44
\def\p[#1]_#2{
	\setbox0=\hbox{$\scriptstyle{#2}$}
	\setbox2=\hbox{$\displaystyle{#1}$}
	\setbox4=\hbox{${}'\mathsurround=0pt$}
	\dimen0=.5\wd0 \advance\dimen0 by-.5\wd2
	\ifdim\dimen0>0pt
	\ifdim\dimen0>\wd4 \kern\wd4 \else\kern\dimen0\fi\fi
	\mathop{{#1}'}_{\kern-\wd4 #2}
}
\def\sump_#1{\p[\sum]_{#1}}
\def\prodp_#1{\p[\prod]_{#1}}
\def\minp_#1{\p[\min]_{#1}}
\def\maxp_#1{\p[\max]_{#1}}
% sets
\newcommand*{\NN}{\mathbb{N}}
\newcommand*{\NNO}{\mathbb{N}_0}
\newcommand*{\ZZ}{\mathbb{Z}}
\newcommand*{\QQ}{\mathbb{Q}}
\newcommand*{\RR}{\mathbb{R}}
\newcommand*{\CC}{\mathbb{C}}
\newcommand*{\CCi}{\mathbb{C}_\infty}
\newcommand*{\FF}{\mathbb{F}}
\newcommand*{\PP}{\mathbb{P}}
\newcommand*{\LL}{\mathcal{L}}
\renewcommand{\AA}{\mathbb{A}}
\newcommand*{\cJ}{\mathcal{J}}
\newcommand*{\finadele}{\AA_F^{fin}}
\newcommand*{\invertadele}{\bigl(\finadele\bigr)^{\!\times}}
\newcommand*{\fp}{\mathfrak{p}}
% \newcommand*{\WeakMF}{\mathfrak{W}}
\DeclareMathOperator{\WeakMF}{\mathbf{Weak}}
\newcommand*{\StrongMF}{\mathbf{Strong}}
\newcommand*{\CuspMF}{\mathbf{Cusp}}
\newcommand*{\cw}{\mathcal{W}}
\newcommand*{\cm}{\mathcal{M}}
\newcommand*{\cs}{\mathcal{S}}
\newcommand*{\LLi}[2]{\protect\overleftarrow{\LL_{#1}^{#2}}}
\newcommand*{\LLNRi}{\LLi{N}{r}}
% arrows
% \newcommand*{\isoto}{\overset{\sim}{\to}}
\newcommand*{\isoto}{\xrightarrow{\mathmakebox[1.5ex]{\sim}}}
\newcommand*{\into}{\hookrightarrow}
\newcommand*{\onto}{\twoheadrightarrow}
\newcommand*{\ionto}{\lhook\joinrel\twoheadrightarrow}
\newcommand*{\xinto}[2][]{\xhookrightarrow[#1]{#2}}
\newcommand*{\xonto}[2][]{\xrightarrow[#1]{#2}\mathrel{\mkern-14mu}\rightarrow}
\newcommand*{\xionto}[2][]{\xhookrightarrow[#1]{#2}\mathrel{\mkern-22mu}\rightarrow\mathrel{\mkern6mu}}
\newcommand*{\mapsfrom}{\mathrel{\reflectbox{\ensuremath{\mapsto}}}}
% maths operators
\DeclareMathOperator{\End}{End}
\DeclareMathOperator{\B}{B}
\DeclareMathOperator{\D}{D}
\DeclareMathOperator{\Cl}{Cl}
\let\Im=\relax \DeclareMathOperator{\Im}{Im}
\DeclareMathOperator{\Sur}{Sur}
\DeclareMathOperator{\Inj}{Inj}
\DeclareMathOperator{\Free}{Free}
\DeclareMathOperator{\GLL}{GL}
\newcommand*{\GL}[2][]{\GLL_{#1}\parens{#2}}
\DeclareMathOperator{\Div}{Div}
\DeclareMathOperator{\bd}{\partial}
\DeclareMathOperator{\ord}{ord}
% differentiation
\newcommand*{\dd}{\operatorname{d}}
\newcommand*{\id}[1]{\,\dd{#1}}
\newcommand*{\dbd}[2][]{\frac{\dd^{#1}}{\dd\!#2^{#1}}}
% enclosures
% \mleftright	% fixes spacing around left-(middle-)right delimiters
% \delimiterfactor=851
% \delimitershortfall=10pt
\newcommand*{\set}[1]{\left\{#1\right\}}
\newcommand*{\setst}[2]{\left\{#1\:\middle|\:#2\right\}}
\newcommand*{\order}[1]{\!O\!\left(#1\right)\!}
\newcommand*{\norm}[1]{\left\lVert 1\right\rVert}
\newcommand*{\abs}[1]{\left\lvert#1\right\rvert}
\newcommand*{\floor}[1]{\left\lfloor#1\right\rfloor}
\newcommand*{\ceil}[1]{\left\lceil#1\right\rceil}
\newcommand*{\fracpart}[1]{\left\{#1\right\}}
\newcommand*{\parens}[1]{\left(#1\right)}
\newcommand*{\pparens}[1]{\parens{\!\parens{#1}\!}}
\newcommand*{\brackets}[1]{\left[#1\right]}
\newcommand*{\braces}[1]{\left\{#1\right\}}
\newcommand*{\size}{\#}
\newcommand*{\sizep}[1]{\size\parens{#1}}
\newcommand*{\sizeset}[1]{\size\set{#1}}
\newcommand*{\sizesetst}[2]{\size\setst{#1}{#2}}
\newcommand*{\lquotient}[2]{\left. #1 \middle\backslash #2 \right.}
\newcommand*{\rquotient}[2]{\left. #1 \middle/ #2 \right.}
\newcommand*{\lrquotient}[3]{\left. #1 \middle\backslash #2 \middle/ #3 \right.}
\newcommand*{\cin}[2]{\bigl[{#1}^{#2}\bigr]}
\newcommand*{\bigO}[1]{\mathcal{O}\parens{#1}}
\newcommand*{\littleo}[1]{o\parens{#1}}
\newcommand*{\pcoord}[1]{%
  \begingroup\lccode`~=`: \lowercase{\endgroup
  \edef~}{\mathbin{\mathchar\the\mathcode`:}\nobreak}%
  \left(% opening symbol
  \begingroup
  \mathcode`:=\string"8000
  #1%
  \endgroup
  \right)% closing symbol
}
% maths text spacing
\newcommand*{\lrsptext}[1]{\quad\text{#1}\quad}
\newcommand*{\lsptext}[1]{\quad\text{#1}\ }
\newcommand*{\rsptext}[1]{\ \text{#1}\quad}
\newcommand*{\sptext}[1]{\ \text{#1}\ }
% other
\newcommand*{\ie}{i.e.\ }
\newcommand*{\eqtag}{\stepcounter{equation} \tag{\theequation}}
\newcommand*{\defeq}{\coloneqq}
\newcommand*{\eqdef}{\eqqcolon}


\title{The Generation of Rank 2 Drinfeld Modular Forms by Eisenstein Series: A Computational Approach}
\author{Liam Baker}
\date{\today}

\begin{document} \maketitle

\begin{abstract}
  A computational method is presented for determining whether or not the space of Drinfeld modular forms for a principal congruence subgroup $\Gamma(N)$ is generated by Eisenstein series, which is currently an open question.
  This method boils down to expressing the dimension of the weight $2$ space generated by Eisenstein series as the rank of a matrix of coefficients of expansions of products of Eisenstein series at the cusps.
  As a result, the question is answered in the case of $q = 2$ and $N$ quadratic or cubic and in the case of $q = 3$ and $N$ quadratic.
\end{abstract}


\section{Introduction and Outline} \label{sec:intro}

\subsection{Outline of the Paper} \label{ssec:outline}

In \cref{sec:intro}, we give an outline of the paper, a review of the work published so far on this specific topic, and a description of the main objects in our playing field.
Then in \cref{sec:algorithm} we describe the algorithm which is the main subject of this paper, together with proof of its validity and bounds on its computational complexity.
In the subsequent \cref{sec:refinements} we present some refinements to the algorithm which speed up its execution; this section also includes some previously unseen relations between Eisenstein series which may even be of interest outside of the context of this algorithm.
Finally in \cref{sec:conclusion} we present the results of the algorithm applied to small values of $N$ and present some concluding remarks.


\subsection{Review of the Relevant Literature} \label{ssec:review}

Drinfeld modules are analogous to elliptic curves in the function field setting, and were first defined by <Drinfeld>, who called them \emph{elliptic modules} and used them to prove <Langlands>.
Later, <Goss> defined what he called \emph{Drinfeld modular forms} of rank $2$ analogously with classical modular forms (in this paper, we will call Drinfeld modular forms simply modular forms, and explicitly mention when their classical counterparts are named).
These functions form a graded algebra graded by their \emph{weight}.
Goss also imposed a condition of `boundedness at infinity' which allowed him to show the finite-dimensionality of the spaces of D-modular forms of any given weight.
<Gekeler> defined Eisenstein series as cases of modular forms, and showed that Eisenstein series of higher weight are generated by Eisenstein series of weight $1$.

The question then naturally arose of whether or not the space of all modular forms is generated as an algebra by the Eisenstein series of weight $1$; this was answered in the affirmative by <Cornellissen> for principal congruence subgroups $\Gamma(N)$ where $N$ is linear, but the question for general $N$ is still open.
The most significant progress thus far has been by <someone>, who showed that the space of all modular forms is generated by the Eisenstein series of weight $1$, possibly together with some cusp forms of weight $2$ (these are modular forms which are zero at all the cusps).
It thus suffices to establish whether or not the Eisenstein series of weight $1$ generates the space of modular forms of weight $2$, and in this paper we give a method for doing so for any specific characteristic $q$ and principal congruence subgroup $\Gamma(N)$, as well as an implementation of this method in SageMath.


\subsection{The Mathematical Setup} \label{ssec:msetup}
Let $q$ be a prime power, and let $\FF_q$ denote the finite field of cardinality $q$.
Let $A = \FF_q[T]$ denote the ring of polynomials with coefficients in $\FF_q$, with $F = \FF_q(T)$ its field of fractions; these can be thought of in analogy with the integers $\ZZ$ and rational numbers $\QQ$ respectively.
$A$ can be equipped with an absolute value $\abs{a} = q^{\deg_T a}$ which is extended to $F$ in the natural way; when $F$ is completed with respect to this absolute value we obtain the field $F_\infty = \FF_q((T))$ of Laurent series in $T$ with finitely many positive powers of $T$, which is analogous to the real numbers $\RR$.
Finally, the completion of an algebraic closure of $F_\infty$ is denoted by $\CCi$; this is analogous to the complex numbers $\CC$, and is where most of the action will happen.

$\CCi$ has rigid analytic structure, as does $\Omega = \CCi - F_\infty$, and as such we can speak of rigid analytic functions on them.
Moreover, the group $\GL[2]{A}$ acts on $\Omega$ on the left by fractional linear transformations $\gamma z = \frac{az+b}{cz+d}$ for $\gamma = \begin{psmallmatrix} a & b \\ c & d \end{psmallmatrix} \in \GL[2]{A}$ and $z \in \CCi$.
That this is a group action can be seen in that it arises from the action of $\GL[2]{A}$ as matrices multiplying on the left of $\PP^2(\CCi)$ with rational lines removed, seen as column vectors: $\begin{psmallmatrix} a & b \\ c & d \end{psmallmatrix} \begin{psmallmatrix} z \\ 1 \end{psmallmatrix} = \begin{psmallmatrix} a z + b \\ c z + d \end{psmallmatrix} = \begin{psmallmatrix} \parens{a z + b}/\parens{c z + d} \\ 1 \end{psmallmatrix}$.
% Indeed, it can be helpful to see $\CCi$ as a subset of $\PP^2(\CCi)$.

As special subgroups of $\GL[2]{A}$ we consider the full congruence subgroups
\[ \Gamma(N) = \ker\parens{\GL[2]{A} \onto \GL[2]{A/N}} = \setst{\gamma \in \GL[2]{A}}{\gamma \equiv \begin{psmallmatrix} 1 & 0 \\ 0 & 1 \end{psmallmatrix} \pmod{N}} \]
where $N$ is an ideal of $A$, or equivalently a monic polynomial since $A = \FF_q[T]$ is a principal ideal domain.

The quotient of $\lquotient{\Gamma(N)}{\Omega}$ is compactified by adding finitely many points $\lquotient{\Gamma(N)}{F}$; these points are called the \emph{cusps} of the compactified space, which is denoted by $\overline{M}_{\Gamma(N)}$.

% introduce the exponential function e_\Lambda(z)

For a nonnegative integer $k$, % introduce modular forms


\section{The Algorithm} \label{sec:algorithm}


\section{Refinements To The Algorithm} \label{sec:refinements}


\section{Conclusion} \label{sec:conclusion}

Due to computational constraints we are only able to run the code for some small nonlinear values of $N$, but we hope that this progress will inspire others to improve on our method or given the evidence above to revisit this interesting question with fresh eyes.

\end{document}